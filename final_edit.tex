\documentclass[]{article}
\usepackage{lmodern}
\usepackage{amssymb,amsmath}
\usepackage{ifxetex,ifluatex}
\usepackage{fixltx2e} % provides \textsubscript
\ifnum 0\ifxetex 1\fi\ifluatex 1\fi=0 % if pdftex
  \usepackage[T1]{fontenc}
  \usepackage[utf8]{inputenc}
\else % if luatex or xelatex
  \ifxetex
    \usepackage{mathspec}
    \usepackage{xltxtra,xunicode}
  \else
    \usepackage{fontspec}
  \fi
  \defaultfontfeatures{Mapping=tex-text,Scale=MatchLowercase}
  \newcommand{\euro}{€}
\fi
% use upquote if available, for straight quotes in verbatim environments
\IfFileExists{upquote.sty}{\usepackage{upquote}}{}
% use microtype if available
\IfFileExists{microtype.sty}{%
\usepackage{microtype}
\UseMicrotypeSet[protrusion]{basicmath} % disable protrusion for tt fonts
}{}
\usepackage[margin=1in]{geometry}
\usepackage{graphicx}
\makeatletter
\def\maxwidth{\ifdim\Gin@nat@width>\linewidth\linewidth\else\Gin@nat@width\fi}
\def\maxheight{\ifdim\Gin@nat@height>\textheight\textheight\else\Gin@nat@height\fi}
\makeatother
% Scale images if necessary, so that they will not overflow the page
% margins by default, and it is still possible to overwrite the defaults
% using explicit options in \includegraphics[width, height, ...]{}
\setkeys{Gin}{width=\maxwidth,height=\maxheight,keepaspectratio}
\ifxetex
  \usepackage[setpagesize=false, % page size defined by xetex
              unicode=false, % unicode breaks when used with xetex
              xetex]{hyperref}
\else
  \usepackage[unicode=true]{hyperref}
\fi
\hypersetup{breaklinks=true,
            bookmarks=true,
            pdfauthor={Michael Shyne},
            pdftitle={Self-care Final Report},
            colorlinks=true,
            citecolor=blue,
            urlcolor=blue,
            linkcolor=magenta,
            pdfborder={0 0 0}}
\urlstyle{same}  % don't use monospace font for urls
\setlength{\parindent}{0pt}
\setlength{\parskip}{6pt plus 2pt minus 1pt}
\setlength{\emergencystretch}{3em}  % prevent overfull lines
\setcounter{secnumdepth}{0}

%%% Use protect on footnotes to avoid problems with footnotes in titles
\let\rmarkdownfootnote\footnote%
\def\footnote{\protect\rmarkdownfootnote}

%%% Change title format to be more compact
\usepackage{titling}

% Create subtitle command for use in maketitle
\newcommand{\subtitle}[1]{
  \posttitle{
    \begin{center}\large#1\end{center}
    }
}

\setlength{\droptitle}{-2em}
  \title{Self-care Final Report}
  \pretitle{\vspace{\droptitle}\centering\huge}
  \posttitle{\par}
  \author{Michael Shyne}
  \preauthor{\centering\large\emph}
  \postauthor{\par}
  \predate{\centering\large\emph}
  \postdate{\par}
  \date{8/14/2015}



\begin{document}

\maketitle


\section{Method}\label{method}

\subsection{Design and Procedure}\label{design-and-procedure}

The study was a treatment-control design, involving working nurses
taking a graduate level nursing theory course at an urban university.
Four sections of the course, conducted at two locations, were randomly
assigned to one of two groups, A or B. All students were asked to
complete a self-care assessment (Health-Promoting Lifestyle Profile II)
at the beginning of the semester and again at the end of the semester,
along with self-identified demographic information. The assessments were
anonymous. ``Pre'' and ``post'' assessments were matched using secret
codes, a short word or number selected by each student. All students
were also instructed on the importance of self-care and asked to write a
reflective summary of their self-care activities. In addition, students
in group A were asked to create and implement a self-care plan based on
the findings of the initial assessment.

\subsection{Instrument}\label{instrument}

The Health-Promoting Lifestyle Profile II (HPLP II) consists of 52
Likert scale questions with 4 possible answers, never (N), sometimes
(S), often (O) and routinely (R). The answers are scored 1 to 4,
respectively. The HPLP II generates seven scores, an overall score and
categorical scores for health responsibility, physical activity,
nutrition, spiritual growth, interpersonal relations and stress
management. The scores are calculated by taking the mean of all answers
for the overall score or the mean of a specified collection of answers
for each category score.

\subsection{Data Entry and Processing}\label{data-entry-and-processing}

At the time of data entry, assessments were matched according to secret
codes. Assessments with nearly matching codes and similar demographic
information and handwriting were included. Six pairs of assessments did
not have matching or nearly matching codes, but were determined to be
likely pairs due to demographic information and handwriting. These were
labeled ``section 3'' for convenience. All initial tests were conducted
with and without ``section 3''. As there was no significant difference
whether ``section 3'' was included, those assessments were included in
the sample for final analysis. One assessment was only half completed
and thus that student was excluded from the sample.

Data from paper assessments were manually entered into a spreadsheet. A
randomly selected 10\% validation sample was checked for errors. No
errors were found.

The seven HPLP II scores (overall and six categories) were calculated
for ``pre'' and ``post'' assessments. The change from ``pre'' scores to
the ``post'' scores, referred to hereafter as delta scores, were
generated for each student. The race variable was simplified into
``white'', ``black'' or ``other'' (which included Hispanic, Indian and
biracial entries).

\subsection{Statistical Analysis}\label{statistical-analysis}

Statistical analysis was conducted using R (version 3.2.0).

Summary statistics were generated for demographic information and HPLP
II scores, including mean, standard deviation, minimum and maximum for
continuous variables and distribution for categorical variables. The
demographic information was examined for differences between groups
using independent t-tests for continuous variables (age and years RN)
and Fisher's test for categorical variables (gender and race). Variables
that showed significant differences between groups (years RN and race)
were tested for effects on HPLP II delta scores, linear regression for
years RN and ANOVA for race.

One sample one-sided t-tests were conducted by group and category to
test whether delta scores were greater than zero. Independent sample
one-sided t-tests were conducted by category to test whether group A
delta scores were greater than group B delta scores.

All tests were conducted at \(\alpha = 0.05\) level of significance.

\section{Results}\label{results}

\subsection{Sample Characteristics}\label{sample-characteristics}

The sample consisted of 70 students, 39 in group A and 31 in group B
(Table\(~\)\ref{tab:demo}). The majority of students were female
(78.6\%) and white (62.9\%). The mean student age was 32.06 years
(sd=6.62) and the mean years working as an RN was 5.14 (sd = 6.06).

\begin{table}[hbt]
\centering
\caption{Sample Demographics}
\begin{tabular}{l r | c c c}
&& Total & Group A & Group B\\
\hline
& n & 70 & 39 & 31\\
&&\\
Age && \\
& Mean (SD)
  & 32.06 (6.62) 
  & 32.243 (6.65)
  & 31.833 (6.7) \\
& Min, Max (Med)
  & 22, 55 (31) 
  & 22, 55 (31)
  & 23, 51 (30) \\
&& \\
Years RN &&\\
& Mean (SD)
  & 5.141 (6.06) 
  & 6.951 (6.96)
  & 2.789 (3.55) \\
& Min, Max (Med)
  & 0, 37 (3) 
  & 0, 37 (5)
  & 0, 15 (1.25) \\
&& \\
Gender &&\\
& Female 
  & 55 (83.3\%)
  & 30 (78.9\%)
  & 25 (89.3\%)\\
& Male 
  & 11 (16.7\%)
  & 8 (21.1\%)
  & 3 (10.7\%)\\
& Missing
  & 4
  & 1
  & 3\\
&&\\
Race &&\\
& White
  & 44 (65.7\%)
  & 27 (71.1\%)
  & 17 (58.6\%)\\
& Black
  & 15 (22.4\%)
  & 10 (26.3\%)
  & 5 (17.2\%)\\
& Other
  & 8 (11.9\%)
  & 1 (2.6\%)
  & 7 (24.1\%)\\
& Missing
  & 3
  & 1
  & 2\\
\end{tabular}
    
\label{tab:demo}
\end{table}

Age (p=0.803) and gender (p=0.331) showed no significant differences
between groups. Years RN (p=0.002) and race (p=0.032) showed significant
differences between groups. Years RN and race were tested for effects on
delta scores using linear regression (Years RN) or ANOVA (Race).
However, no significant effect on delta scores was found for either
variable (Table\(~\)\ref{tab:effect}).

\begin{table}[hbt]
\centering
\caption{Effect of Years RN and Race on category scores}

\begin{tabular}{r|cccc}
& \multicolumn{2}{c}{Years RN} & \multicolumn{2}{c}{Race}\\

  & t & p &  F & p \\ 
  \hline
Overall & -0.500 & 0.618 & 2.457 & 0.094 \\ 
  Health Responsibility & -0.642 & 0.523 & 2.074 & 0.134 \\ 
  Physical Activity & -0.747 & 0.458 & 1.652 & 0.200 \\ 
  Nutrition & -0.111 & 0.912 & 1.139 & 0.326 \\ 
  Spiritual Growth & 0.410 & 0.683 & 1.817 & 0.171 \\ 
  Interpersonal Relations & 0.092 & 0.927 & 2.176 & 0.122 \\ 
  Stress Management & -0.894 & 0.374 & 0.639 & 0.531 \\ 
  \end{tabular}
    
\label{tab:effect}
\end{table}

\subsection{Score Characteristics}\label{score-characteristics}

Overall scores for ``pre'' assessments ranged from 1.83 to 3.63 with a
mean of 2.76 (sd=0.37). ``Post'' assessment scores ranged from 1.83 to
3.6 with a mean of 2.9 (sd=0.42). Delta scores ranged from -0.56 to 0.92
with a mean of 0.14 (sd=0.31) (Table \ref{tab:score}). Figure
\ref{fig:score} displays mean scores and 95\% confidence intervals (from
appropriate t distributions) by group and category for ``pre'', ``post''
and ``delta'' scores.

\begin{table}[hbt]
\centering
\caption{Score Characteristics}
\begin{tabular}{l c  p{0.001cm} | c c c }
&&& Total & Group A & Group B \\
\hline
\multicolumn{2}{c}{ Overall } &&\\ Pre  & Mean (SD) &  & 2.76 (0.37) & 2.77 (0.37) & 2.74 (0.38) \\& Med (Min, Max) &  & 2.78 (1.83, 3.63) & 2.79 (1.9, 3.57) & 2.75 (1.83, 3.63) \\Post  & Mean (SD) &  & 2.9 (0.42) & 2.96 (0.41) & 2.84 (0.44) \\& Med (Min, Max) &  & 2.89 (1.83, 3.6) & 2.94 (1.98, 3.56) & 2.81 (1.83, 3.6) \\Delta  & Mean (SD) &  & 0.14 (0.31) & 0.18 (0.3) & 0.09 (0.31) \\& Med (Min, Max) &  & 0.13 (-0.56, 0.92) & 0.13 (-0.4, 0.92) & 0.13 (-0.56, 0.48) \\&&\\\multicolumn{2}{c}{ Health Responsibility } &&\\ Pre  & Mean (SD) &  & 2.53 (0.45) & 2.53 (0.49) & 2.53 (0.4) \\& Med (Min, Max) &  & 2.56 (1.44, 3.67) & 2.56 (1.44, 3.67) & 2.56 (1.56, 3.44) \\Post  & Mean (SD) &  & 2.76 (0.51) & 2.77 (0.51) & 2.74 (0.51) \\& Med (Min, Max) &  & 2.83 (1.75, 3.89) & 2.78 (1.78, 3.67) & 2.89 (1.75, 3.89) \\Delta  & Mean (SD) &  & 0.23 (0.43) & 0.24 (0.43) & 0.21 (0.45) \\& Med (Min, Max) &  & 0.22 (-1, 1.22) & 0.22 (-1, 1.22) & 0.22 (-0.58, 1.11) \\&&\\\multicolumn{2}{c}{ Physical Activity } &&\\ Pre  & Mean (SD) &  & 2.52 (0.68) & 2.57 (0.72) & 2.47 (0.64) \\& Med (Min, Max) &  & 2.38 (1.25, 3.88) & 2.38 (1.25, 3.88) & 2.38 (1.5, 3.88) \\Post  & Mean (SD) &  & 2.63 (0.68) & 2.67 (0.67) & 2.57 (0.71) \\& Med (Min, Max) &  & 2.6 (1.12, 4) & 2.62 (1.38, 4) & 2.5 (1.12, 3.75) \\Delta  & Mean (SD) &  & 0.1 (0.52) & 0.11 (0.58) & 0.1 (0.44) \\& Med (Min, Max) &  & 0.06 (-0.88, 1.38) & 0 (-0.88, 1.25) & 0.12 (-0.88, 1.38) \\&&\\\multicolumn{2}{c}{ Nutrition } &&\\ Pre  & Mean (SD) &  & 2.7 (0.48) & 2.67 (0.49) & 2.73 (0.47) \\& Med (Min, Max) &  & 2.67 (1.56, 3.78) & 2.67 (1.67, 3.78) & 2.67 (1.56, 3.67) \\Post  & Mean (SD) &  & 2.86 (0.5) & 2.89 (0.52) & 2.81 (0.48) \\& Med (Min, Max) &  & 2.89 (1.56, 3.89) & 3 (1.56, 3.89) & 2.78 (1.89, 3.67) \\Delta  & Mean (SD) &  & 0.16 (0.36) & 0.22 (0.4) & 0.07 (0.29) \\& Med (Min, Max) &  & 0.22 (-0.78, 0.89) & 0.22 (-0.78, 0.89) & 0 (-0.56, 0.67) \\&&\\\multicolumn{2}{c}{ Spiritual Growth } &&\\ Pre  & Mean (SD) &  & 3.17 (0.44) & 3.2 (0.45) & 3.13 (0.42) \\& Med (Min, Max) &  & 3.22 (2.11, 4) & 3.33 (2.22, 4) & 3.11 (2.11, 4) \\Post  & Mean (SD) &  & 3.29 (0.51) & 3.4 (0.47) & 3.15 (0.54) \\& Med (Min, Max) &  & 3.33 (1.89, 4) & 3.56 (2.33, 4) & 3.22 (1.89, 4) \\Delta  & Mean (SD) &  & 0.11 (0.38) & 0.19 (0.36) & 0.01 (0.39) \\& Med (Min, Max) &  & 0.11 (-0.78, 1.11) & 0.11 (-0.56, 1.11) & 0 (-0.78, 0.78) \\&&\\\multicolumn{2}{c}{ Interpersonal Relations } &&\\ Pre  & Mean (SD) &  & 3.1 (0.46) & 3.16 (0.49) & 3.03 (0.42) \\& Med (Min, Max) &  & 3.11 (1.89, 4) & 3.11 (1.89, 4) & 3 (2.22, 4) \\Post  & Mean (SD) &  & 3.26 (0.52) & 3.34 (0.47) & 3.15 (0.56) \\& Med (Min, Max) &  & 3.33 (1.78, 4) & 3.44 (2, 4) & 3.11 (1.78, 4) \\Delta  & Mean (SD) &  & 0.16 (0.37) & 0.18 (0.32) & 0.13 (0.44) \\& Med (Min, Max) &  & 0.22 (-0.67, 1) & 0.22 (-0.56, 0.78) & 0.22 (-0.67, 1) \\&&\\\multicolumn{2}{c}{ Stress Management } &&\\ Pre  & Mean (SD) &  & 2.48 (0.47) & 2.44 (0.41) & 2.52 (0.55) \\& Med (Min, Max) &  & 2.46 (1.25, 3.75) & 2.38 (1.75, 3.25) & 2.5 (1.25, 3.75) \\Post  & Mean (SD) &  & 2.57 (0.55) & 2.58 (0.48) & 2.55 (0.63) \\& Med (Min, Max) &  & 2.62 (1.38, 4) & 2.62 (1.62, 3.62) & 2.62 (1.38, 4) \\Delta  & Mean (SD) &  & 0.09 (0.5) & 0.14 (0.49) & 0.04 (0.52) \\& Med (Min, Max) &  & 0.12 (-1.12, 1.38) & 0.12 (-0.75, 1.38) & 0.12 (-1.12, 1.12) \\&&\\
\end{tabular}
    
\label{tab:score}
\end{table}

\begin{figure} [hbt]
\caption{Mean scores by group and category with 95\% confidence intervals}
\includegraphics{final_files/figure-latex/unnamed-chunk-8-1.pdf}
\label{fig:score}
\end{figure}

\subsection{Group Effect on Scores}\label{group-effect-on-scores}

Group A delta scores were significantly greater than zero for all
categories except physical activity (Table \ref{tab:tests}). Group B scores were not
significantly greater than zero in any category except health
responsibility. Group A delta scores were significantly greater than
group B scores in the categories nutrition (p=0.035) and spiritual
growth (p=0.027).

\begin{table}[hbt] \centering
\caption{One-sided t-tests of delta scores with lower
bound of 95\% confidence interval}

\begin{tabular}{r|ccccccccc}
& \multicolumn{3}{c}{A>0} & \multicolumn{3}{c}{B>0} & \multicolumn{3}{c}{A>B (A-B>0)}\\
  & t & p & lower & t & p & lower & t & p & lower \\ 
  \hline
Overall & 3.805 & 0.000 & 0.102 & 1.681 & 0.052 & -0.001 & 1.206 & 0.116 & -0.034 \\ 
  Health Responsibility & 3.536 & 0.001 & 0.126 & 2.624 & 0.007 & 0.075 & 0.275 & 0.392 & -0.148 \\ 
  Physical Activity & 1.186 & 0.121 & -0.046 & 1.228 & 0.114 & -0.037 & 0.104 & 0.459 & -0.190 \\ 
  Nutrition & 3.507 & 0.001 & 0.116 & 1.361 & 0.092 & -0.018 & 1.837 & 0.035 & 0.014 \\ 
  Spiritual Growth & 3.350 & 0.001 & 0.096 & 0.197 & 0.423 & -0.106 & 1.973 & 0.027 & 0.028 \\ 
  Interpersonal Relations & 3.584 & 0.000 & 0.096 & 1.599 & 0.060 & -0.008 & 0.602 & 0.275 & -0.100 \\ 
  Stress Management & 1.754 & 0.044 & 0.005 & 0.385 & 0.351 & -0.124 & 0.820 & 0.208 & -0.104 \\ 
  \end{tabular}

\label{tab:tests} 
\end{table}

\section{Conclusion}\label{conclusion}

Mean scores for every category and each group showed positive change
from ``pre'' assessments to ``post'' assessments. However, only for
group A were a majority of the scores (including the overall score)
significantly larger than zero. The health responsibility category
showed the greatest growth with a mean change of 0.23 (sd=0.43) while
stress management had the lowest with a mean change of 0.09 (sd=0.5).

\end{document}
